%% This is file `DEMO-TUDaSciPoster.tex' version 2.09 (2020/03/13),
%% it is part of
%% TUDa-CI -- Corporate Design for TU Darmstadt
%% ----------------------------------------------------------------------------
%%
%%  Copyright (C) 2018--2020 by Marei Peischl <marei@peitex.de>
%%
%% ============================================================================
%% This work may be distributed and/or modified under the
%% conditions of the LaTeX Project Public License, either version 1.3c
%% of this license or (at your option) any later version.
%% The latest version of this license is in
%% http://www.latex-project.org/lppl.txt
%% and version 1.3c or later is part of all distributions of LaTeX
%% version 2008/05/04 or later.
%%
%% This work has the LPPL maintenance status `maintained'.
%%
%% The Current Maintainers of this work are
%%   Marei Peischl <tuda-ci@peitex.de>
%%   Markus Lazanowski <latex@ce.tu-darmstadt.de>
%%
%% The development respository can be found at
%% https://github.com/tudace/tuda_latex_templates
%% Please use the issue tracker for feedback!
%%
%% ============================================================================
%%
% !TeX program = lualatex
%%

\documentclass[
	accentcolor=9c,
%	boxstyle= boxed, % Boxen mit abgerundeten Ecken, farbigem Titelblock
%	boxstyle=colored % Boxen mit farbigen Titelblock, keine vertikalen Linien
%	boxstyle=default % Voreinstellung, ohne Farbe, ohne vertikale Linien
%	logofile=example-image, %Falls die Logo Dateien nicht vorliegen
	]{tudasciposter}

%Sprache
\usepackage[ngerman]{babel}
\usepackage{microtype}
\usepackage[autostyle]{csquotes}


%Makros für dieses Beispieldokument. Im Allgemeinen nicht notwendig.
\newcommand{\tbs}{\textbackslash}
\newcommand{\repl}[1]{<\textit{#1}>}
\let\code\texttt
\newcommand*{\macro}[1]{\code{\tbs#1}}
\let\file\texttt
\let\pck\textsf
\let\cls\textsf


\begin{document}
\title{\pck{tcolorbox}-Poster im Corporate Design der TU~Darmstadt}
\author{Marei Peischl\thanks{kontakt@peitex.de} \and der \TeX-Löwe}
\institute{pei\TeX{} \TeX{}nical Solutions, Regensburg}
\titlegraphic{\includegraphics[width=.5\linewidth]{example-image}}%TODO größe
\footerqrcode{https://www.peitex.de}
\footer{Fußzeile: Falls neben den Logos andere Informationen erforderlich sind}

%Instituts/Sponsorenlogos
\footergraphics{
	\includegraphics[height=\height ]{example-image}
	\includegraphics[height=\height ]{example-image}
	\includegraphics[height=\height ]{example-image}
}

\begin{tcbposter}[
	poster={
		columns=4,
		rows=7,
		spacing=1cm,
%		showframe, %Gitter einblenden. Für Platzierung häufig hilfreich
	},]

	\begin{posterboxenv}[title=Zusammenfassung]{name=intro,column=1,row=1,span=4}
	Die \cls{tudasciposter}-Klasse basiert auf dem \pck{tcolorbox} Paket von Thomas F. Sturm.
	Sie versucht einen einfachen Weg zu bieten, wissenschaftliche Poster im Corporate Design der TU Darmstadt zu erstellen. Dieses Dokument dient als Dokumentation und Verwendungsbeispiel.

	Dieses Dokument verwendet unterschiedliche Boxentypen. Dies ist selbstverständlich für die praktische Verwendung nicht empfehlenswert. Dieser Modus dient lediglich Demonstrationszwecken.
	\end{posterboxenv}

	\begin{posterboxenv}[title=Titelei]{name=title, row=2, span=2,rowspan=2}
	Die Definition des Titelblockes funktioniert analog zu Standard-\LaTeX{} mit \macro{maketitle}.

	Für die Datenübergabe stehen die Makros \macro{title}, \macro{author}, \macro{institute} und \macro{titlegraphic} zur Verfügung. Letztere wird rechtsbündig  unterhalb des TUDa-Logos platziert. Die \macro{linewidth} zu diesem Zeitpunkt entspricht der Breite des TUDa-Logos.

	Zusätzlich zu den Titeldaten stehen über \macro{setqrcode} und \macro{setfoot} Makros zur Verfügung, die die Fußzeile füllen.
	Ein Beispiel ist in der Datei \file{DEMO"=TUDaSciPoster.tex} gezeigt
	\end{posterboxenv}

\begin{posterboxenv}[title=Fußzeile]{name=footer,below=title, span=2, rowspan=2 }
	Die Fußzeile ist grundsätzlich aktiviert, kann jedoch über die Klassenoption \code{footer=false} deaktiviert werden. In diesem Fall werden jedoch mit \macro{thanks} übergebene Zusätzliche Titelinformationen nicht angezeigt.

	Für die Übergabe weiterer Daten stehen die Makros \macro{footer}, \macro{footergraphics} und \macro{footerqrcode} zur Verfügung.

	\macro{footergraphics} ist für die Übergabe von Logos gedacht und \macro{footerqrcode} übernimmt eine URL die anschließend in der Rechten unteren Ecke als QRCode platziert wird.

	Die Fußzeile selbst erhält die Daten aus \macro{thanks}, kann jedoch ergänzt werden. Sie hat die Breite des Satzspiegels abzüglich der Logos/QRcode.
\end{posterboxenv}


\begin{posterboxenv}[title=Platzierung der Boxen]{name=positioning,below=footer, span=2}
Bei der \pck{poster}-Bibliothek des \pck{tcolorbox} Paketes, werden die Boxen manuell positioniert.

Dies benötigt zwar einen zusätzlichen Arbeitsschritt, erlaubt jedoch einer feinere Ausrichtung der Boxen, auch relativ zueinander.

Diese Mechanismen ermöglichen Auch Querverweise einfacher zu positionieren. Hierfür ist ein Blick in die \pck{tcolorbox}-Dokumentation hilfreich,
\end{posterboxenv}

\begin{posterboxenv}[title=Zusammenfassung]{name=intro,column=1,row=1,span=4}
	Die \cls{tudasciposter}-Klasse basiert auf dem \pck{tcolorbox} Paket von Thomas F. Sturm.
	Sie versucht einen einfachen Weg zu bieten, wissenschaftliche Poster im Corporate Design der TU Darmstadt zu erstellen. Dieses Dokument dient als Dokumentation und Verwendungsbeispiel.

	Dieses Dokument verwendet unterschiedliche Boxentypen. Dies ist selbstverständlich für die praktische Verwendung nicht empfehlenswert. Dieser Modus dient lediglich Demonstrationszwecken.
\end{posterboxenv}

\begin{posterboxenv}[title=Titelei]{name=title, row=2, span=2,rowspan=2}
	Die Definition des Titelblockes funktioniert analog zu Standard-\LaTeX{} mit \macro{maketitle}.

	Für die Datenübergabe stehen die Makros \macro{title}, \macro{author}, \macro{institute} und \macro{titlegraphic} zur Verfügung. Letztere wird rechtsbündig  unterhalb des TUDa-Logos platziert. Die \macro{linewidth} zu diesem Zeitpunkt entspricht der Breite des TUDa-Logos.

	Zusätzlich zu den Titeldaten stehen über \macro{setqrcode} und \macro{setfoot} Makros zur Verfügung, die die Fußzeile füllen.
	Ein Beispiel ist in der Datei \file{DEMO"=TUDaSciPoster.tex} gezeigt
\end{posterboxenv}

\begin{posterboxenv}[title=Fußzeile]{name=footer,below=title, span=2, rowspan=2 }
	Die Fußzeile ist grundsätzlich aktiviert, kann jedoch über die Klassenoption \code{footer=false} deaktiviert werden. In diesem Fall werden jedoch mit \macro{thanks} übergebene Zusätzliche Titelinformationen nicht angezeigt.

	Für die Übergabe weiterer Daten stehen die Makros \macro{footer}, \macro{footergraphics} und \macro{footerqrcode} zur Verfügung.

	\macro{footergraphics} ist für die Übergabe von Logos gedacht und \macro{footerqrcode} übernimmt eine URL die anschließend in der Rechten unteren Ecke als QRCode platziert wird.

	Die Fußzeile selbst erhält die Daten aus \macro{thanks}, kann jedoch ergänzt werden. Sie hat die Breite des Satzspiegels abzüglich der Logos/QRcode.
\end{posterboxenv}


\begin{posterboxenv}[title=Platzierung der Boxen]{name=positioning,below=footer, span=2}
	Bei der \pck{poster}-Bibliothek des \pck{tcolorbox} Paketes, werden die Boxen manuell positioniert.

	Dies benötigt zwar einen zusätzlichen Arbeitsschritt, erlaubt jedoch einer feinere Ausrichtung der Boxen, auch relativ zueinander.

	Diese Mechanismen ermöglichen Auch Querverweise einfacher zu positionieren. Hierfür ist ein Blick in die \pck{tcolorbox}-Dokumentation hilfreich,
\end{posterboxenv}

\begin{posterboxenv}[title=Eine Box im Stil TUDa-boxed, TUDa-boxed]{name=boxed,column=3, row=2, span=2}
	Die Boxen können in verschiedenen Varianten Gestaltet werden. Die Voreinstellung entspricht den offiziellen Vorgaben, jedoch kann es aus unterschiedlichen Gründen notwendig sein, eine klarere Abgrenzung zu setzen (lobale Klassenoption \code{boxstyle=boxed} oder lokaler Stil \code{TUDa-boxed}).
\end{posterboxenv}


\begin{posterboxenv}[title=Eine Box im Stil TUDa-colored, TUDa-colored]{name=colored,column=3, row=3, span=2}
	Eine Zwischenstufe zwischen dem \code{boxed} und dem \code{official} Stil stellt dieser Boxentyp dar.

	Einstellung über globale Klassenoption \code{style=colored} oder lokaler Stil \code{TUDa-colored}

\end{posterboxenv}

\begin{posterboxenv}[TUDa-colored]{name=colored-notitle,column=3,row=4, rowspan=2}
	\includegraphics[width=\linewidth]{example-image}
	\captionof{figure}{Ein Beispielbild, in einer Box ohne Titel. In diesem Fall sind der Stil {TUDa} und TUDa-colored identisch}
\end{posterboxenv}

\begin{posterboxenv}[TUDa-boxed]{name=boxed-notitle,column=4, below=colored}
	Ein Beispiel für die den Stil \code{boxed} ohne Titel.
\end{posterboxenv}


\begin{posterboxenv}[title=Box mit Link,TUDa-boxed]{name=relative,column=4, above=row6}
	Beispiel mit Pfeil, um zwei Boxen miteinander zu Verknüpfen oder Leseabzweigungen zu generieren.
\end{posterboxenv}

%Zwischen den Boxen kann direkt TikZ-Code eingegeben werden. Das Namensschema der Boxen als Koordinaten lautet
% TCBPOSTER@<boxname>.ankerpunkt
% Für genauere Erläuterungen zur Syntax, bietet die tikz-Anleitung genauere Informationen, weitere benannte Koordinaten finden sich in der tcolorbox-Dokumentation
\draw[accentcolor,line width=4pt,->] ([yshift=-1cm]TCBPOSTER@relative.east) -|  ([xshift=1cm]TCBPOSTER@colored.east) -- (TCBPOSTER@colored.east);

\begin{posterboxenv}[title=relative Positionierung,TUDa-boxed]{name=relative,column=4, between=boxed-notitle and relative}
	Beispiel für die relative Positionierung, diese Box zwischen zwei Boxen platziert.
\end{posterboxenv}

\begin{posterboxenv}[title=Papierformat]{name=paper,column=3,span=2,below=row5}
Die Klasse \cls{tudasciposter} unterstützt die Papierformate A0, A1, A2 und A3. Der Wert wird über die Klassenoption \code{paper} ausgewählt:

\begin{verbatim}
paper=a0
\end{verbatim}
Die Voreinstellung entspricht \code{a0}.
Die Änderung des Papierformates ist keine Skalierung, da Schriftgrößen nicht direkt skalieren.

Um eine Skalierung eines größeren auf ein kleineres Designs zu erreichen, empfiehlt es sich das Ausgangsformat beim Druck zu skalieren oder ggf. die PDF-Datei mit Paketen wie \pck{pdfpages} oder eine PDF-Drucker umzurechnen.
\end{posterboxenv}

\end{tcbposter}

\end{document}


